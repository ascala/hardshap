%% LyX 2.2.3 created this file.  For more info, see http://www.lyx.org/.
%% Do not edit unless you really know what you are doing.
\documentclass[english]{article}
\usepackage[T1]{fontenc}
\usepackage[latin9]{inputenc}
\usepackage{babel}
\begin{document}

\title{Ellipses}
\maketitle

\section{Class Ellipse}
\begin{itemize}
\item $r$ center of mass
\item $u[0]$,$u[1]$ are the $x$,$y$ axis
\item $a_{x}$,$a_{y}$ are the axis lengths
\end{itemize}

\subsection{Useful Functions}

The surface is described by 

\[
f\left(P\right)=\left[\frac{\left(P-r\right)\cdot u[0]}{a_{x}}\right]^{2}+\left[\frac{\left(P-r\right)\cdot u[1]}{a_{y}}\right]^{2}-1=0
\]

can described implicitly in terms of a single parameter (angle) $\alpha$:

\[
P\left(\alpha\right)=a_{x}\,\cos\alpha\,u[0]+a_{y}\sin\alpha\,u[1]+r
\]

\[
\partial_{\alpha}P\left(\alpha\right)=-a_{x}\,\sin\alpha\,u[0]+a_{y}\cos\alpha\,u[1]
\]

The gradient $\nabla f\left(P\right)$ is 

\[
\nabla f\left(P\right)=2\frac{\left(P-r\right)\cdot u[0]}{a_{x}^{2}}\,u[0]+2\frac{\left(P-r\right)\cdot u[1]}{a_{y}^{2}}\,u[1]
\]

In the case $P\left(\alpha\right)$, we have $\left(P-r\right)\cdot u[0]=a_{x}\,\cos\alpha$

\[
\nabla f\left(P\left(\alpha\right)\right)=2\cos\alpha\,u[0]/a_{x}+2\sin\alpha\,u[1]/a_{y}
\]

that can be used to check that $\partial_{\alpha}f\left(P\left(\alpha\right)\right)=0$

\[
\partial_{\alpha}f\left(P\left(\alpha\right)\right)=\nabla f\cdot\partial_{\alpha}P=0
\]

and to calculate $\partial_{\alpha}\nabla f\left(P\left(\alpha\right)\right)$

\[
\partial_{\alpha}\nabla f\left(P\left(\alpha\right)\right)=-2\sin\alpha\,u[0]/a_{x}+2\cos\alpha\,u[1]/a_{y}
\]

\section{Distance}

We will look for the distance between two ellipses using Newton-Raphson.
Given two ellipses $E_{i}\,(i=A,B)$, we will individuate two points
$P_{i}$ on the surfaces of $E_{i}$ (i.e. corresponding to two angles
$\theta_{i}$) such that

\[
\left\{ \begin{array}{ccc}
\nabla_{A}f_{A}\wedge\nabla_{B}f_{B} & = & 0\\
P_{B} & = & P_{A}+\alpha\nabla_{A}f_{A}
\end{array}\right.
\]

i.e. we look for the $0$'s of $3$ equations (as $P_{i}=\left(x_{i},y_{i}\right)$)
:

\[
F=\left(\begin{array}{c}
\nabla_{A}f_{A}\wedge\nabla_{B}f_{B}\\
x_{A}-x_{B}+\alpha\left.\nabla_{A}f_{A}\right|_{x}\\
y_{A}-y_{B}+\alpha\left.\nabla_{A}f_{A}\right|_{y}
\end{array}\right)=0
\]

with Hessian

\[
H=\left\Vert \begin{array}{ccc}
\left(\partial_{\theta_{A}}\nabla_{A}f_{A}\right)\wedge\nabla_{B}f_{B} & \nabla_{A}f_{A}\wedge\left(\partial_{\theta_{B}}\nabla_{B}f_{B}\right) & 0\\
\partial_{\theta_{A}}x_{A}+\alpha\left(\partial_{\theta_{A}}\nabla_{A}f_{A}\right)_{x} & -\partial_{\theta_{B}}x_{B} & \left.\nabla_{A}f_{A}\right|_{x}\\
\partial_{\theta_{A}}y_{A}+\alpha\left(\partial_{\theta_{A}}\nabla_{A}f_{A}\right)_{y} & -\partial_{\theta_{B}}y_{b} & \left.\nabla_{A}f_{A}\right|_{y}
\end{array}\right\Vert 
\]

Therefore, we have all the routines for the Hessian 

\section{Class Ellipse}
\begin{itemize}
\item $r$ center of mass
\item $u[0]$,$u[1]$ are the $x$,$y$ axis
\item $a_{x}$,$a_{y}$ are the axis lengths
\item $e_{x}$,$e_{y}$ are the $x$,$y$ exponents 
\end{itemize}

\subsection{Useful Functions}

The surface is described by 

\[
f\left(P\right)=\left[\frac{\left(P-r\right)\cdot u[0]}{a_{x}}\right]^{e_{x}}+\left[\frac{\left(P-r\right)\cdot u[1]}{a_{y}}\right]^{e_{y}}-1=0
\]

(where for $a^{e}$ we really mean $sgn\left(a\right)\left|a\right|^{e}$)
and can described implicitly in terms of a single parameter (angle)
$\alpha$:

\[
P\left(\alpha\right)=a_{x}s_{x}\left|\cos\alpha\right|^{2/e_{x}}\,u[0]+a_{y}s_{y}\left|\sin\theta\right|^{2/e_{y}}\,u[1]+r
\]

\[
\partial_{\alpha}P\left(\alpha\right)=-\left(2a_{x}/e_{x}\right)s_{x}\left|\sin\alpha\right|\,\left|\cos\alpha\right|^{\left(2-e_{x}\right)/e_{x}}\,u[0]+\left(2a_{y}/e_{y}\right)s_{y}\left|\cos\alpha\right|\,\left(\sin\alpha\right)^{\left(2-e_{x}\right)/e_{y}}\,u[1]
\]

where $s_{x}=sgn\left(\cos\alpha\right)$, $s_{y}=sgn\left(\sin\alpha\right)$

The gradient $\nabla f\left(P\right)$ is 

\[
\nabla f\left(P\right)=2\frac{\left(P-r\right)\cdot u[0]}{a_{x}^{2}}\,u[0]+2\frac{\left(P-r\right)\cdot u[1]}{a_{y}^{2}}\,u[1]
\]

In the case $P\left(\alpha\right)$, we have $\left(P-r\right)\cdot u[0]=a_{x}\,\cos\alpha$

\[
\nabla f\left(P\left(\alpha\right)\right)=2\cos\alpha\,u[0]/a_{x}+2\sin\alpha\,u[1]/a_{y}
\]

that can be used to check that $\partial_{\alpha}f\left(P\left(\alpha\right)\right)=0$

\[
\partial_{\alpha}f\left(P\left(\alpha\right)\right)=\nabla f\cdot\partial_{\alpha}P=0
\]

and to calculate $\partial_{\alpha}\nabla f\left(P\left(\alpha\right)\right)$

\[
\partial_{\alpha}\nabla f\left(P\left(\alpha\right)\right)=-2\sin\alpha\,u[0]/a_{x}+2\cos\alpha\,u[1]/a_{y}
\]


\end{document}
